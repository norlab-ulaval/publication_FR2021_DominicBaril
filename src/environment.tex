\section{Environment}
\label{sec:env}

The main focus of this work is deploy our \ac{LTR} framework in a subarctic environment and evaluate it's performance when subject to complex weather conditions.
In order to do so, we conducted our deployments within the \textit{Montmorency} boreal forest, located \SI{70}{km} North of \quebec~city, Canada.
Characteristics of boreal forests include dense closed-crown conifer vegetation~\citep{Russell1988} and long, cold winters leading to a high snow cover on the ground.
Dense vegetation is known to cause an issue for real-time mapping due to static antenna signal being blocked for \ac{RTK} \ac{GNSS}~\citep{Babin2019}.
Snow cover, illumination variation and the uniformity of the environment are also known to be challenging for vision-based approaches~\citep{Paton2017}.
Three different one-way paths were defined, all link 2 \acp{POI}.
The A  and B paths link the Garage and \laverdiere~\acp{POI} through cross-country ski trails, measuring \SI{1.5}{km} and \SI{1.5}{km} respectively.
In order to maximize \ac{UGV} and prevent immobilization, the A and B paths was previously driven on by a snowmobile operator.
The C path connects the Garage and Gazebo \acp{POI} mostly in a larger road network, measuring \SI{0.6}{km}.
The path network is illustrated in~\autoref{fig:forest}.

\begin{figure}[htpb]
	\begin{center}
		\begin{subfigure}[b]{0.45\textwidth}
			\includegraphics[width=\linewidth]{figs/map-dem.pdf}
			%\caption{A view from above the \textit{Montmorency} forest, which covers \SI{400}{km^2}.}
			\label{fig:view_above}
		\end{subfigure}%
		~~
		\begin{subfigure}[b]{0.45\textwidth}
			\includegraphics[width=\linewidth]{figs/foret-montmorency-path.pdf}
			%\caption{An example of a snowy path, in which our system performed \ac{LTR}.}
			\label{fig:view_path}
		\end{subfigure}%
		%% Maybe change with a camera image?
		\caption{A digital terrain model of the area where our \ac{LTR} system was deployed.
		We see the three different paths, both the A (\SI{1.5}{km}) and B path ({1.5}{km}) going from the garage \ac{POI} to the \laverdiere~\ac{POI}, while the C path (\SI{0.6}{km}) goes to the Gazebo \ac{POI}.
		The image on the right is an example of the A and B paths, which mostly consists of cross-country sky trails.
		Image credit : \foretmo.} 
		\label{fig:forest}
	\end{center}
\end{figure}

During the five days deployment, we completed
% Discuss weather

\begin{figure} [htpb]
	\centering
	\includegraphics[height=2.0in]{example-image}
	\caption{Johann's various runs and meteo figure}
	\label{fig:meteo_runs}
\end{figure}

\begin{table}[htpb]
	\caption{Overview of all runs realized in this work. 
		All times are defined in the local Eastern Standard Time. 
		The column $\Delta t$ defines the elapsed time since the teach run of the associated path.} \label{tab:all_runs}
	\begin{center}
		\begin{tabular}{c c c c c c}
			\hline
			% after \\: \hline or \cline{col1-col2} \cline{col3-col4} ...
			ID & Path & Start time & Duration & $\Delta t$ & Number of scans \\
			\hline
			TeachA & A & \DTMdate{2021-03-30} \DTMtime{11:04:00} & \DTMtime{00:25:00} & Teach Pass \\
			TeachB & B' & \DTMdate{2021-03-29} \DTMtime{15:45:00} & \DTMtime{00:31:00} & Teach Pass \\
			TeachC & C & \DTMdate{2021-03-30} \DTMtime{07:28:00} & \DTMtime{00:15:00} & Teach Pass \\
			R1 & C & \DTMdate{2021-03-31} \DTMtime{10:42:00} & \DTMtime{00:22:00} & \DTMtime{27:14:00} &  \\
			R2 & B & \DTMdate{2021-03-31} \DTMtime{14:03:00} & \DTMtime{00:26:00} & \DTMtime{30:35:00} &  \\
			R3 & A' & \DTMdate{2021-03-31} \DTMtime{15:02:00} & \DTMtime{00:27:00} & \DTMtime{31:34:00} &  \\
			R4 & A & \DTMdate{2021-03-31} \DTMtime{20:42:00} & \DTMtime{00:26:00} & \DTMtime{37:14:00} &  \\
			R5 & B' & \DTMdate{2021-03-31} \DTMtime{21:12:00} & \DTMtime{00:33:00} & \DTMtime{37:44:00} &  \\
			R6 & B & \DTMdate{2021-03-31} \DTMtime{22:00:00} & \DTMtime{00:35:00} & \DTMtime{38:32:00} &  \\
			R7 & A' & \DTMdate{2021-03-31} \DTMtime{22:47:00} & Battery breakdown & \DTMtime{39:19:00} &  \\
			R8 & A & \DTMdate{2021-04-01} \DTMtime{09:21:00} & \DTMtime{00:16:00} & \DTMtime{49:53:00} &  \\
			R9 & B' & \DTMdate{2021-04-01} \DTMtime{10:19:00} & \DTMtime{00:29:00} & \DTMtime{50:51:00} &  \\
			R10 & C & \DTMdate{2021-04-01} \DTMtime{11:01:00} & \DTMtime{00:23:00} & \DTMtime{51:33:00} &  \\
			R11 & B & \DTMdate{2021-04-01} \DTMtime{18:13:00} & \DTMtime{00:23:00} & \DTMtime{58:45:00} &  \\
			R12 & A' & \DTMdate{2021-04-01} \DTMtime{19:09:00} & \DTMtime{00:00:00} & \DTMtime{59:41:00} &  \\
			R13 & A & \DTMdate{2021-04-02} \DTMtime{06:53:00} & \DTMtime{00:26:00} & \DTMtime{71:25:00} &  \\
			R14 & B' & \DTMdate{2021-04-02} \DTMtime{07:25:00} & \DTMtime{00:29:00} & \DTMtime{71:57:00} &  \\
			
		\end{tabular}
	\end{center}
\end{table}

% Validate run durations through bag files

% Number of points per maps : 
%Map A: 1729505 points
%Map B: 2423769 points
%Map C: 637690 points

\lightlipsum[1]