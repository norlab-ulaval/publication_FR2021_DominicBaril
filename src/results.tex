\section{Results}
\label{sec:results}

The goal of this section is to quantify the performance of our \ac{LTR} framework when deployed in the environment described in~\autoref{sec:env}.
We achieve this by evaluating the localization accuracy of the \ac{ICP} algorithm through various metrics.
We also show our camera and \ac{GNSS} measurements to demonstrate navigation approaches using these sensors as primary mean of localization would suffer significant performance loss in a subarctic forest.
Afterwards, we evaluate the path-following performance of our controller and quantify our \ac{UGV}'s energy consumption for each run.
Lastly, we identify failure cases for our system and they were handled in the field.

\subsection{Localization}
\label{sec:res_loc}

\lightlipsum[1]

\subsubsection{Vision-based}
\label{sec:res_vis}

\lightlipsum[1]

\begin{figure} [htpb]
	\centering
	\includegraphics[height=2.0in]{example-image}
	\caption{Olivier's over and under exposition figure for cameras}
	\label{fig:cameras_expo}
\end{figure}

\subsubsection{GNSS}
\label{sec:res_gnss}

\lightlipsum[1]

\begin{figure} [htpb]
	\centering
	\includegraphics[height=4.0in]{./figs/GPS/FR_gps_data_fails2.pdf}
	\caption{Examples of GPS positioning error along the path A and B.}
	\label{fig:gnss_error_path}
\end{figure}

\begin{figure} [htpb]
	\centering
	\includegraphics[height=2.0in]{./figs/GPS/RTK_error.pdf}
	\caption{GNSS error for each runs.}
	\label{fig:gnss_run_error}
\end{figure}

\begin{figure} [htpb]
	\centering
	\includegraphics[height=2.0in]{./figs/GPS/Satellite_number.pdf}
	\caption{GNSS satellite number for each run.}
	\label{fig:gnss_satellite_number}
\end{figure}

\subsubsection{ICP}
\label{sec:ICP}

\lightlipsum[1]

\begin{figure} [htpb]
	\centering
	\includegraphics[height=2.0in]{example-image}
	\caption{Figure explaining ICP error for every run (correlated with meteo).}
	\label{fig:icp_error}
\end{figure}



\subsection{Motion and control}
\label{sec:res_motion}

In order to characterize the performance of our path-following controller, we computed the cross track error for each measured position in each repeat run.
Our definition of the cross-track error is the distance between the robot frame $\robotf$'s origin and it's orthogonal projection on the path, as defined in~\citep{Mondoloni2005}.
The results for the cross-track error are displayed in~\autoref{fig:pf_error}.
It can be seen that for all trajectories, the cross-track error mostly remains below \SI{0.1}{m} for most of the runs.
Path curvature can be correlated with an increased cross-track error, with a maximum observed error of \SI{0.8}{m}.
In terms of different runs, it can be observed that cross-track error is stable, despite varying meteorological conditions. 

\begin{figure}[htpb]
	\begin{center}
		\begin{subfigure}[b]{\textwidth}
			\centering
			\includegraphics[height=2.5in]{figs/ref_traj_errors.pdf}
			\caption{Mean cross-track error with respect to the reference trajectories.}
			\label{fig:pf_error_traj}
		\end{subfigure}
		\\
		\begin{subfigure}[b]{\textwidth}
			\centering
			\includegraphics[height=1.6in]{figs/pf_error_runs.pdf}
			\caption{Cross-track error for all runs.}
			\label{fig:pf_error_runs}
		\end{subfigure}
		\caption{Cross-track error during the deployment.
		The cross-track error was computed for every localization position recorded during each repeat run.
		For each map, the coordinates are defined in the reference map frame $\mapf$, which are the same as shown in~\autoref{fig:ref_ltr}.} 
		\label{fig:pf_error}
	\end{center}
\end{figure}


\begin{figure} [htpb]
	\centering
	\includegraphics[height=2.0in]{example-image}
	\caption{Power consumption / motion efficiency figure.}
	\label{fig:moiton_power}
\end{figure}

\subsection{Failure cases}
\label{sec:fail}
%% Discuss Laverdiere failure and garage failure

\subsubsection{Run 2 initialization}
\label{sec:laverdiere_fail}

\lightlipsum[1]

\subsubsection{Run 10 failed initialization}
\label{sec:laverdiere_fail}

\lightlipsum[1]

\begin{figure} [htpb]
	\centering
	\includegraphics[height=1.8in]{example-image}
	\caption{Figure explaining special cases when mapping needed to be enabled.}
	\label{fig:icp_failure}
\end{figure}