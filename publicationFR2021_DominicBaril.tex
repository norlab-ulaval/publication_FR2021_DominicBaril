% !TeX root = publicationFR2021_DominicBaril.tex
% !TEX program = latexmk
% !TEX encoding = UTF-8 Unicode
% !TEX options = --shell-escape -synctex=1 -interaction=nonstopmode -file-line-error -pdf
% vim: ft=tex


\documentclass{article}
\usepackage{frExamplee, times}
%%\usepackage{graphicx}
%%\usepackage{apalike}
\usepackage{setspace}
\input{./latexGoodPractices/preamble}
\addbibresource{./references.bib}

% Add subfigure
\usepackage{subcaption}

% Add figure with caption at the side
\usepackage{sidecap}

% For dates and time
\usepackage{datetime2}
\DTMnewtimestyle{custom}{
	\renewcommand{\DTMdisplaytime}[3]{
		\DTMtexorpdfstring{\DTMtwodigits{##1}:\DTMtwodigits{##2}}
	}
}
\DTMsettimestyle{custom}

%For tikz figure
\usepackage{tikz}
\usepackage{pgfplots}
\usepackage{csvsimple}
\usetikzlibrary{shapes}

% For checkmarks
\usepackage{pifont}% http://ctan.org/pkg/pifont
\newcommand{\cmark}{\ding{51}}%
\newcommand{\xmark}{\ding{55}}%


% Text commands
\newcommand{\laverdiere}{Laverdi\`{e}re} % Laverdière
\newcommand{\quebec}{Qu\'{e}bec} % Québec
\newcommand{\foretmo}{For\^{e}t Montmorency} % Québec

% Useful math commands
\newcommand{\mapf}{\mathcal{G}} % map reference frame
\newcommand{\odomf}{\mathcal{O}} % odom reference frame
\newcommand{\robotf}{\mathcal{R}} % robot reference frame
\newcommand{\lidarf}{\mathcal{L}} % lidar reference frame
\newcommand{\pathf}{\mathcal{S}} % path Frenet-Serret frame
\newcommand{\skreadpc}{\check{\mathcal{P}}} % Skewed reading point cloud
\newcommand{\readpc}{\mathcal{P}} % reading point cloud
\newcommand{\refpc}{\mathcal{Q}} % reference point cloud
\newcommand{\match}{\mathcal{M}} % set of point matches
\newcommand{\weight}{\mathcal{W}} % set of point weights
\newcommand{\transform}[2]{$_{#1}^{#2}\bm{T}$} % Transform from #1 frame to #2 frame
\newcommand{\reftraj}{\bm x_{\text{ref}}} % Reference trajectory
\newcommand{\poseplane}{\bm x_{\text{2D}}} % 2D robot pose

% argmin and argmax
\DeclareMathOperator*{\argmin}{arg\,min}

%JL: is there a specific reason for this notation? I find the notation _^{R1}T_{R2} somewhat better, but it might be personal!


\title{Lidar-based teach-and-repeat in subarctic conditions}

\author{
Dominic Baril\thanks{ Use footnote for providing further information
	about author (webpage, alternative address). Acknowledgments to
	funding agencies should go in the \textbf{Acknowledgments} section
	at the end of the paper.} \\
Norlab, Universit\'{e} Laval\\
Qu\'{e}bec, QC, Canada G1V 0A6 \\
\texttt{dominic.baril@norlab.ulaval.ca} \\
\And
Simon-Pierre Desch\^{e}nes \\
Norlab, Universit\'{e} Laval\\
Qu\'{e}bec, QC, Canada G1V 0A6 \\
\And
Olivier Gamache \\
Norlab, Universit\'{e} Laval\\
Qu\'{e}bec, QC, Canada G1V 0A6 \\
\And
Maxime Vaidis \\
Norlab, Universit\'{e} Laval\\
Qu\'{e}bec, QC, Canada G1V 0A6 \\
\And
Damien LaRocque \\
Norlab, Universit\'{e} Laval\\
Qu\'{e}bec, QC, Canada G1V 0A6 \\
\And
Johann Laconte \\
Norlab, Universit\'{e} Laval\\
Qu\'{e}bec, QC, Canada G1V 0A6 \\
\And
Vladim\'{i}r Kubelka \\
Norlab, Universit\'{e} Laval\\
Qu\'{e}bec, QC, Canada G1V 0A6 \\
\And
Philippe Gigu\`{e}re \\
Norlab, Universit\'{e} Laval\\
Qu\'{e}bec, QC, Canada G1V 0A6 \\
\And
Fran\c{c}ois Pomerleau \\
Norlab, Universit\'{e} Laval\\
Qu\'{e}bec, QC, Canada G1V 0A6 \\
\texttt{francois.pomerleau@norlab.ulaval.ca} \\
}

% The \author macro works with any number of authors. There are two commands
% used to separate the names and addresses of multiple authors: \And and \AND.
%
% Using \And between authors leaves it to \LaTeX{} to determine where to break
% the lines. Using \AND forces a linebreak at that point. So, if \LaTeX{}
% puts 3 of 4 authors names on the first line, and the last on the second
% line, try using \AND instead of \And before the third author name.

% Acronym definitions
\acrodef{SNOW}{Self-driving Navigation Optimized for Winter}
\acrodef{SSMR}{Skid-steering mobile robot}
\acrodef{UGV}{unmanned ground vehicle}
\acrodef{GPR}{ground penetrating radar}
\acrodef{IMU}{inertial measurement unit}
\acrodef{GPS}{Global Positioning System}
\acrodef{RTK}{Real-time Kinematics}
\acrodef{GNSS}{Global Navigation Satellite System}
\acrodef{ICP}{iterative closest point}
\acrodef{SLAM}{Simultaneous Localization and Mapping}
\acrodef{DoF}{degree of freedom}
\acrodef{ICR}{instantaneous centre of rotation}
\acrodef{ROC}{radius of curvature}
\acrodef{COM}{center of mass}
\acrodef{ROS}{Robot Operating System}
\acrodef{MPC}{Model Predictive Control}
\acrodef{VTR}[VT\&R]{Visual Teach and Repeat}
\acrodef{LTR}[LT\&R]{Lidar Teach and Repeat}
\acrodef{EBN}{Experience-based navigation}
\acrodef{MEL}{Multi-experience Localization}
\acrodef{CNN}{Convolutional Neural Network}
\acrodef{GeRoNa}{Generic Robot Navigation}
\acrodef{ORTHEXP}{Orthogonal-Exponential}
\acrodef{DD-ORTHEXP}{Differential-Drive-ORTHEXP}
\acrodef{POI}{Points of Interest}

\begin{document}

\maketitle

\begin{abstract}
	\lightlipsum[1]
\end{abstract}

\section{Introduction}
\label{sec:intro}

\lightlipsum[1]

\begin{figure} [h]
	\centering
	\includegraphics[height=2.0in]{example-image}
	\caption{Warthog driving / Aerial shot of the different paths}
	\label{fig:front_fig}
\end{figure}

\lightlipsum[1]
\lightlipsum[1]
\lightlipsum[1]
\section{Related work}
\label{sec:rel_work}

\subsection{Robotic deployments in snow}
\label{sec:snow_robots}

To our knowledge, few robots have been deployed in harsh winter environments. 
Dante II is a \SI{900}{kg} tethered legged robot, which conducted a 5-day, \SI{165}{m} descent into the Mount Spurr Volcano, in Alaska~\citep{Bares1999}.
During this deployment, Dante II reached speeds upwards to \SI{0.011}{m/s} during the descent.
A two-axis lidar was used to create a local elevation map around the robot in order to conduct autonomous navigation.

Nomad is a gasoline-powered \SI{725}{kg} \ac{UGV}, was deployed at Elephant Moraine, Antarctica for a duration of 4 weeks~\citep{Apostolopoulos2000}. 
The robot reached speeds upwards of \SI{0.5}{m/s} while using differential-\ac{GPS} as the primary method of localization.
The platform also used stereo cameras and a lidar sensor for obstacle detection, although stereo vision was found to be ineffective  on blue ice and snow in Antarctica due to extreme lack of texture~\citep{Moorehead1999}.
Roll/pitch/yaw sensors were also added to the robot to make it cognizant to hazardous terrain.
Nomad achieved its initial goal to identify meteorites autonomously in Antarctica at a search rate of ~  \SI{160}{m^2/h}.

MARVIN I and MARVIN II are two diesel-powered \acp{SSMR} weighing \SI{720}{kg} were deployed in Greenland~\citep{Stansbury2004} and Antarctica~\citep{Gifford2009} respectively. 
The goal of these robots was to increase survey safety in remote polar regions and large sensor payloads led to the selection of large vehicles.
Both vehicles used \ac{RTK} \ac{GPS} as primary method, achieving a centimeter-level accuracy.
They also used a lidar sensor for obstacle detection and a a gyroscope and inclinometer were used to provide the robot's pitch and roll angles.
Skid-steer turns often caused MARVIN I to get immobilized in snow and its transmission eventually broke down during operation.
MARVIN II thus incorporated design improvements to the hydro-static drive and track systems to increase its durability.


Sno-mote Mk1 and Mk2 are dual-drive 1:10 scale snowmobiles equipped with a single camera and \ac{GPS} antenna were deployed on Alaskan glaciers and Wapekoneta, Ohio~\citep{Williams2009}.
These robots were used to conduct manually-driven traverses of about \SI{100}{m} at a speed of ~\SI{1}{m/s}.
The data gathered with the Sno-motes was then used to improve visual \ac{SLAM} feature extraction methods in snow. 
Despite improving feature detection methods on snow, it was shown that snow is still feature-sparse~\citep{Williams2009}.
Through this work, improvements were also done on slope estimation~\citep{Williams2010} and horizon line estimation~\citep{Williams2011}.

Yeti is a battery-powered \SI{81}{kg} \ac{UGV} in Antarctica and Greenland~\citep{Lever2013}. 
Yeti was used to conduct \ac{GPR} surveys in order to detect subsurface crevasses or other voids to increase vehicle travel safety in remote polar environments.
Since polar terrain is largely obstacle-free and the effort required to provide reliable obstacle detection on low-contrast snowfields is considerable, Yeti drove "blind", relying only on \ac{GPS} waypoint following.
During surveys, Yeti reached a top speed of \SI{2.2}{m/s} and managed to acquire data on hundreds of crevasse encounters and even locate a previously undetected buried building in the South Pole.

A Clearpath Robotics Grizzly, a battery- and gasoline-powered \ac{SSMR} was deployed during winter on the University of Toronto Institute for Aerospace Studies (UTIAS) campus, in Ontario, Canada~\citep{Paton2017}.
Only stereo cameras were used through a visual \ac{SLAM} algorithm to localize the robot during autonomous teach-and-repeat runs. 
Path tracking was accomplished using a \ac{MPC} algorithm.
A 250 m path was successfully repeated on an light snow cover 3 hours after it was first manually driven. 
However, deep snow path-following provided unsatisfactory results due to features almost only being observed on the horizon, leading to inaccurate pose estimates, which caused issues for the path tracker.
Furthermore, vehicle tracks that constantly change when driven over lead to an increased pose estimation error.

A full-scale battery-powered Toyota Prius was deployed during winter on roads in Massachusetts, USA~\citep{Ort2020}.
Localization was accomplished using a custom-designed localizing \ac{GPR}. 
A prior mapping must be conducted during which the driven is driven by a human operator and the vehicle's sensor data is recorded, the saved map can then allow the vehicle to localize within this area.
The \ac{GPR} location information is then probabilistically fused with wheel odometry and \ac{IMU} measurements to provide accurate vehicle localization.
Path tracking is accomplished through the use of a Pure Pursuit controller, specifically designed for Ackermann steered autonomous vehicles.
The system showed similar performance in localization accuracy (\SI{0.34}{m} to \SI{0.39}{m}) and cross-track error (\SI{0.26}{m} to \SI{0.29}{m}) between clear weather and snow-covered road.
The localizing \ac{GPR} sensor's measurement range depends on the width of the array, meaning the system cannot be easily miniaturized, which means it was mounted on the rear of the vehicle, at \SI{32}{cm} above the ground.
This sensor size and mounting requirement could lead to decreased performance in deep snow or in off-road environments. 

\subsection{Relative navigation}
\label{rel_nav}



% UofT TnR and other relevant
\lightlipsum[1]
\lightlipsum[1]
\section{System description}
\label{sec:sys}

\lightlipsum[1]


\subsection{Hardware description}
\label{sec:hardware}

\lightlipsum[1]

\begin{figure} [htpb]
	\centering
	\includegraphics[height=2.0in]{example-image}
	\caption{Warthog figure, pointing to every sensor.}
	\label{fig:warthog}
\end{figure}

\subsection{Lidar teach-and-repeat}
\label{sec:LTR}
\lightlipsum[1]

\begin{figure} [htpb]
	\centering
	\includegraphics[height=2.0in]{example-image}
	\caption{Flowchart for LTR}
	\label{fig:ltr_flow}
\end{figure}

\subsubsection{Iterative closest point}
\label{ICP}

\lightlipsum[1]

\begin{figure} [htpb]
	\centering
	\includegraphics[height=2.0in]{example-image}
	\caption{Figure explaining Simon-Pierre's tiled mapping framework}
	\label{fig:tiled_map}
\end{figure}

\subsubsection{Path following}
\label{sec:orthexp}

\lightlipsum[1]

\begin{figure} [htpb]
	\centering
	\includegraphics[height=2.0in]{example-image}
	\caption{Figure explaining Differential orthogonal-exponential controller}
	\label{fig:diff_orthexp}
\end{figure}
\section{Environment}
\label{sec:env}

\lightlipsum[1]

\begin{figure} [htpb]
	\centering
	\includegraphics[height=2.0in]{example-image}
	\caption{Johann's various runs and meteo figure}
	\label{fig:meteo_runs}
\end{figure}

\lightlipsum[1]
\section{Results}
\label{sec:results}

The goal of this section is to quantify the performance of our \ac{LTR} framework when deployed in the environment described in~\autoref{sec:env}.
We achieve this by evaluating the localization accuracy of the \ac{ICP} algorithm through various metrics.
We also show our camera and \ac{GNSS} measurements to demonstrate navigation approaches using these sensors as primary mean of localization would suffer significant performance loss in a subarctic forest.
Afterwards, we evaluate the path-following performance of our controller and quantify our \ac{UGV}'s energy consumption for each run.
Lastly, we identify failure cases for our system and they were handled in the field.

\subsection{Localization}
\label{sec:res_loc}

\lightlipsum[1]

\subsubsection{Vision-based}
\label{sec:res_vis}

%\lightlipsum[1]
\todo{Should I add a section on the camera's parameters?}
\todo{I am not sure if the police of each column of the figure is the good one for the article...}

The Dalsa C1920 camera installed on the Warthog can be use in low temperature which allow us to analyze the feasibility to use this type of sensor in a subarctic forest. One of the biggest problem related to photography in winter conditions is caused by the reflection of light by snow which often leads to an over exposition and then to the saturation of the image. In the left column of~\autoref{fig:cameras_expo} there is an example where the robot was driving in the wood (bottom left) and an other sequence from the same run where the robot was exposed to the sun (top right). The reflection on the path saturates completely the image and removes all the details from the surroundings. In the middle column of the same figure, the pictures were taken during a snowfall. The snowflakes are practically invisible in the images, which means that they don't affect the sensor. The camera was also tested at night to analyze the possibility to use it at every moment of the day. Run 1's column demonstrates clearly that the lack of light prevents the camera from working in dark situations.

\begin{figure} [htpb]
	\centering
	\includegraphics[height=3.0in]{figs/figure_camera.pdf}
	\caption{Pictures taken for three different runs. The left column from TeachA is an example of the sun's effect on the quality of the images. The images from the middle column were taken during a snowfall (most prominent snowflakes are circled in red). The right column from R1 was taken at night, hence the lack of luminosity.}
	\label{fig:cameras_expo}
\end{figure}

\subsubsection{GNSS}
\label{sec:res_gnss}

\lightlipsum[1]

\begin{figure} [htpb]
	\centering
	\includegraphics[height=2.0in]{example-image}
	\caption{Maxime's GNSS error figure}
	\label{fig:gnss_error}
\end{figure}

\subsubsection{ICP}
\label{sec:ICP}

\lightlipsum[1]

\begin{figure} [htpb]
	\centering
	\includegraphics[height=2.0in]{example-image}
	\caption{Figure explaining ICP error for every run (correlated with meteo).}
	\label{fig:icp_error}
\end{figure}



\subsection{Motion and control}
\label{sec:res_motion}

In order to characterize the performance of our path-following controller, we computed the cross track error for each measured position in each repeat run.
Our definition of the cross-track error is the distance between the robot frame $\robotf$'s origin and it's orthogonal projection on the path, as defined in~\citep{Mondoloni2005}.
The results for the cross-track error are displayed in~\autoref{fig:pf_error}.
It can be seen that for all trajectories, the cross-track error mostly remains below \SI{0.1}{m} for most of the runs.
Path curvature can be correlated with an increased cross-track error, with a maximum observed error of \SI{0.8}{m}.
In terms of different runs, it can be observed that cross-track error is stable, despite varying meteorological conditions. 

\begin{figure}[htpb]
	\begin{center}
		\begin{subfigure}[b]{\textwidth}
			\centering
			\includegraphics[height=2.5in]{figs/ref_traj_errors.pdf}
			\caption{Mean cross-track error with respect to the reference trajectories.}
			\label{fig:pf_error_traj}
		\end{subfigure}
		\\
		\begin{subfigure}[b]{\textwidth}
			\centering
			\includegraphics[height=1.6in]{figs/pf_error_runs.pdf}
			\caption{Cross-track error for all runs.}
			\label{fig:pf_error_runs}
		\end{subfigure}
		\caption{Cross-track error during the deployment.
		The cross-track error was computed for every localization position recorded during each repeat run.
		For each map, the coordinates are defined in the reference map frame $\mapf$, which are the same as shown in~\autoref{fig:ref_ltr}.} 
		\label{fig:pf_error}
	\end{center}
\end{figure}


\begin{figure} [htpb]
	\centering
	\includegraphics[height=2.0in]{example-image}
	\caption{Power consumption / motion efficiency figure.}
	\label{fig:moiton_power}
\end{figure}

\subsection{Failure cases}
\label{sec:fail}
%% Discuss Laverdiere failure and garage failure

\subsubsection{Run 2 initialization}
\label{sec:laverdiere_fail}

\lightlipsum[1]

\subsubsection{Run 10 failed initialization}
\label{sec:laverdiere_fail}

\lightlipsum[1]

\begin{figure} [htpb]
	\centering
	\includegraphics[height=1.8in]{example-image}
	\caption{Figure explaining special cases when mapping needed to be enabled.}
	\label{fig:icp_failure}
\end{figure}
\section{Lessons learned}
\label{sec:lessons}

\lightlipsum[1]
\lightlipsum[1]

% Forest canyon (parallel to urban canyon)
% Dynamic environment
% Simple controller works despite harsh weather, power consumption unaffected too
% GNSS / vision not viable for deep subarctic forest navigation
\section{Conclusion}
\label{sec:concl}

\lightlipsum[1]

\subsubsection*{Acknowledgments}
This research was supported by the Natural Sciences and Engineering Research Council of Canada (NSERC) through the grant CRDPJ 527642-18 SNOW (Self-driving Navigation Optimized for Winter) and FORAC.

% To add : Damien's dad, meteorological lab



%\bibliographystyle{apalike}
%\bibliography{references.bib}
\printbibliography


\end{document}



